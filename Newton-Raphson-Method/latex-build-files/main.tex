\documentclass{article}
\usepackage[utf8]{inputenc}
\usepackage{amsmath}

\title{Newton Raphson Method}
\author{AJM432}
\date{February 2022}

\begin{document}
\maketitle

Let $g(x)$ denote the tangent line of $f(x)$ at $\alpha$.
Then $$f(\alpha) = \alpha f'(\alpha) + b$$
Therefore $$b=f(\alpha)- \alpha f'(\alpha)$$
$$g(x)=f'(\alpha)x + f(\alpha)- \alpha f'(\alpha)$$

Now we must set $g(x)=0$ to find a value of $\alpha$ closer to a root of $f(x)$.

$$f'(\alpha)x + f(\alpha)- \alpha f'(\alpha)=0$$
$$x=\frac{ \alpha f'(\alpha)-f(\alpha)}{f'(\alpha)}$$
Now we may set this value of $x$ as our next point of iteration.
Now we can input this x back into the tangent equation to get a value closer to a root. $$\boxed{\alpha_{n+1}=\frac{ \alpha_n f'(\alpha_n)-f(\alpha_n)}{f'(\alpha_n)}}$$

Now we may iterate the above equation starting at an arbitrary value for $\alpha$ until $\alpha_{n+1}$ approaches a fixed value.

\end{document}
